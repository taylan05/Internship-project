\documentclass{article}
\usepackage[left=0.8in, right=0.8in, top=0.7in, bottom=0.7in]{geometry} % Réduction des marges
\usepackage{graphicx} % Required for inserting images
\usepackage{amsmath}
\usepackage{amssymb}
\usepackage{amsthm}
\usepackage{listings}
\newtheorem{theorem}{Theorem}[section] % Numérotation par section
\newtheorem{definition}{Definition}[section]
\newtheorem{lemma}{Lemma}[section]
\usepackage{mathtools}
\usepackage{bm}
\usepackage{physics}
\usepackage{calligra}
\usepackage{xcolor}
\usepackage{sectsty}
\usepackage{algorithm}
\usepackage{algpseudocode}
\subsubsectionfont{\centering}


\lstdefinelanguage{Matlab}{
  morekeywords={break,case,catch,continue,elseif,else,end,for,function,
    global,if,otherwise,persistent,return,switch,try,while,zeros,ones,
    length,classdef,properties,methods},
  sensitive=true,
  morecomment=[l]\%,
  morestring=[m]',
}[keywords,comments,strings]

\lstset{
  language=Matlab,
  basicstyle=\ttfamily\small,
  keywordstyle=\color{blue}\bfseries,
  stringstyle=\color{orange},
  commentstyle=\color{green!50!black},
  numbers=left,
  numberstyle=\tiny,
  stepnumber=1,
  numbersep=5pt,
  backgroundcolor=\color{gray!10},
  showspaces=false,
  showstringspaces=false,
  frame=single,
  tabsize=2,
  breaklines=true,
  captionpos=b
}

\lstset{literate=
  {é}{{\'e}}1 {è}{{\`e}}1 {ê}{{\^e}}1 {ë}{{\"e}}1
  {É}{{\'E}}1 {È}{{\`E}}1 {Ê}{{\^E}}1 {Ë}{{\"E}}1
  {à}{{\`a}}1 {â}{{\^a}}1 {ä}{{\"a}}1
  {î}{{\^i}}1 {ï}{{\"i}}1
  {ô}{{\^o}}1 {ö}{{\"o}}1
  {ù}{{\`u}}1 {û}{{\^u}}1 {ü}{{\"u}}1
  {ç}{{\c{c}}}1
  {’}{{'}}1  % apostrophe typographique remplacée par simple '
}


\title{Internship}
\author{Taylan Alpgündüz}
\date{September - January 2025}

\begin{document}

\maketitle

\vspace{2cm}

\section{Collaboration}

\vspace{1cm}

\subsection{GIThub}

\vspace{1cm}

For this internship, I had to create a new repository on my GIThub account. I created a folder structure (code, latex, literature) and added a .gitignore file for each folder (and the main folder.

Here is how it is organized : 

\subsubsection*{Code}


At the moment, two folders have been created in this Code folder : 

\begin{itemize}
    \item Functions : Where I put all the codes for the mathematical tools used (DMD, DMD with control, Operator inference, etc...).
    \item Workflow : Where i put all the rest of the codes needed for this whole project (Data prepocessing, Data compression, Visualization of the results, etc...).
\end{itemize}

\subsubsection*{Latex}


This folder will only contain the .tex file of this report.


\subsubsection*{Literature}


Here are all the ressources I needed to have a better understanding of the project (books, articles, interactive codes, etc...).

\vspace{1cm}


\subsection{Jupyter notebooks}

\vspace{1cm}

I also had to create a new virtual environment for the project, with Anaconda. I had to import lots of packages to run those codes, especially for the operator inference one, where opinf is needed for example.

Here is an example of the requirements needed to run the operator inference code, with the package and the version associated : 

cvxpy = 1.3.2 

cvxopt = 1.3.2

findiff = 0.12.1

ipykernel = 6.29.5

jupytext = 1.17.2

jupyter = 1.1.1 

matplotlib = 3.10.3

meshio = 5.3.5

notebook

numpy = 2.3.0

pyvista = 0.43.9

scipy = 1.15.3

scs = 3.2.3

ipywidgets = 8.1.7


\vspace{1cm}

\section{Mathematical introduction}

\vspace{1cm}

In this section, I will write every mathematical tools needed to work on this project.

\vspace{1cm}

\subsection{The Singular Value Decomposition}

\vspace{1cm}

\begin{theorem}
Let $A \in \mathbb{C}^{n,m}$ with $n \ge m$ be given.  
Then there exist unitary matrices  
$V \in \mathbb{C}^{n,n}$ and $W \in \mathbb{C}^{m,m}$ such that  
\[
A = V \Sigma W^H
\]
with
\[
\Sigma =
\begin{bmatrix}
\Sigma_r & 0_{r,\,m-r} \\
0_{n-r,\,r} & 0_{n-r,\,m-r}
\end{bmatrix}
\in \mathbb{R}^{n,m},
\quad
\Sigma_r = \mathrm{diag}(\sigma_1, \dots, \sigma_r),
\]
where
\[
\sigma_1 \ge \sigma_2 \ge \cdots \ge \sigma_r > 0
\quad\text{and}\quad
r = \mathrm{rank}(A).
\]
\end{theorem}

\vspace{1cm}

As the proof shows, this Theorem can be formulated analogously for real matrices 
$A \in \mathbb{R}^{n,m}$ with $n \geq m$. In this case the two matrices $V$ and $W$ 
are orthogonal. If $n < m$ we can apply the theorem to $A^H$ (resp.\ $A^T$ in the real case).


\vspace{1cm}

\begin{definition}
A decomposition of the form \((19.1)\) is called a \emph{singular value decomposition}  
(or short \emph{SVD}) of the matrix \(A\).  

The diagonal entries of the matrix \(\Sigma_r\) are called \emph{singular values},  
and the columns of \(V\) (resp. \(W\)) are called the \emph{left} (resp. \emph{right})  
singular vectors of \(A\).

From \((19.1)\) we obtain the unitary diagonalizations of the matrices \(A^H A\) and \(A A^H\):
\[
A^H A =
W
\begin{bmatrix}
\Sigma_r^2 & 0 \\
0 & 0
\end{bmatrix}
W^H,
\quad
A A^H =
V
\begin{bmatrix}
\Sigma_r^2 & 0 \\
0 & 0
\end{bmatrix}
V^H.
\]

The singular values of \(A\) are therefore uniquely determined as the positive square roots  
of the positive eigenvalues of \(A^H A\) or \(A A^H\).  
The unitary matrices \(V\) and \(W\) in the singular value decomposition, however, are  
(not unlike eigenvectors in general) not uniquely determined.

If we write the SVD of $A$ in the form (19.1),
\[
A = V \Sigma W^H
= V
\begin{bmatrix}
I_m \\[2pt]
0_{n-m,\,m}
\end{bmatrix}
W^H \;
W
\begin{bmatrix}
\Sigma_r & 0 \\
0 & 0_{m-r}
\end{bmatrix}
W^H
=:\, U P,
\]
then $U \in \mathbb{C}^{n,m}$ has orthonormal columns, i.e., $U^H U = I_m$, and
$P = P^H \in \mathbb{C}^{m,m}$ is positive semidefinite with inertia $(r,0,m-r)$,
where $\Sigma_r = \mathrm{diag}(\sigma_1,\dots,\sigma_r)$ and $r=\mathrm{rank}(A)$.

The factorization $A = U P$ is called a \emph{polar decomposition} of $A$.
It can be viewed as a generalization of the polar representation of complex numbers,
$z = e^{i\varphi}\lvert z\rvert$.

\end{definition}

\vspace{1cm}

\begin{lemma}
Suppose that the matrix $A \in \mathbb{C}^{n,m}$ with $\operatorname{rank}(A)=r$ has an SVD of the form (19.1) with $V = [v_1,\dots,v_n]$ and $W = [w_1,\dots,w_m]$.
Considering $A$ as an element of $\mathcal{L}(\mathbb{C}^{m,1}, \mathbb{C}^{n,1})$, we then have
\[
\operatorname{im}(A) = \operatorname{span}\{v_1,\dots,v_r\}
\quad\text{and}\quad
\ker(A) = \operatorname{span}\{w_{r+1},\dots,w_m\}.
\]
\end{lemma}

\vspace{2cm}

An SVD of the form (19.1) can be written as
\[
A = \sum_{j=1}^r \sigma_j\, v_j w_j^H.
\]
Thus, $A$ can be written as a sum of $r$ matrices of the form $\sigma_j v_j w_j^H$, where
$\operatorname{rank}(\sigma_j v_j w_j^H)=1$. Let
\[
A_k := \sum_{j=1}^k \sigma_j\, v_j w_j^H, \qquad 1 \le k \le r.
\tag{19.2}
\]
Then $\operatorname{rank}(A_k)=k$ and, using that the matrix $2$-norm is unitarily invariant (cf. Exercise 19.1), we get
\[
\|A - A_k\|_2
= \left\| \operatorname{diag}(\sigma_{k+1},\dots,\sigma_r) \right\|_2
= \sigma_{k+1}.
\tag{19.3}
\]
Hence $A$ is approximated by the matrix $A_k$, where the rank of the approximating
matrix and the approximation error in the matrix $2$-norm are explicitly known.
Furthermore, the singular value decomposition yields the best possible approximation
of $A$ by a matrix of rank $k$ with respect to the matrix $2$-norm:
\[
\min_{\operatorname{rank}(B)\le k} \|A - B\|_2 = \sigma_{k+1},
\quad\text{attained by } B = A_k.
\]

\vspace{1cm}

The SVD is one of the most important and practical mathematical tools in almost all areas ofscience, engineering and social sciences,in medicine and even in psychology. Its great importance is due to the fact that the SVD allows to distinguish between “important” and “non-important” information in a given data. In practice, the latter corresponds, e.g., to measurement errors, noise in the transmission of data, or fine details in a signal or an image that do not play an important role. Often, the “important” information corresponds to the large singular values, and the “non-important” information to the small ones.

\vspace{1cm}

Another important application of the SVD arises in the solution of linear systems
of equations. If $A \in \mathbb{C}^{n,m}$ has an SVD of the form (19.1), we define the matrix
\[
A^{\dagger} := W \Sigma^{\dagger} V^H \in \mathbb{C}^{m,n},
\]
where
\[
\Sigma^{\dagger} := 
\begin{bmatrix}
\Sigma_r^{-1} & 0 \\
0 & 0
\end{bmatrix}
\in \mathbb{R}^{m,n}.
\]


One easily sees that
\[
A^{\dagger} A = W 
\begin{bmatrix}
I_r & 0 \\
0 & 0
\end{bmatrix}
W^H \in \mathbb{R}^{m,m}.
\]

If $r = m = n$, then $A$ is invertible and the right-hand side of the above equation is 
equal to the identity matrix $I_n$. In this case we have 
\[
A^{\dagger} = A^{-1}.
\]
The matrix $A^{\dagger}$ can therefore be viewed as a generalized inverse, that in the case 
of an invertible matrix $A$ is equal to the inverse of $A$.

\vspace{1cm}

\begin{definition}

The matrix $A^{\dagger}$ in (19.4) is called the \emph{Moore--Penrose inverse} 
or \emph{pseudoinverse} of $A$.  

Let $A \in \mathbb{C}^{n,m}$ and $b \in \mathbb{C}^{n,1}$ be given. If the linear system 
of equations
\[
Ax = b
\]
has no solution, then we can try to find an $\hat{x} \in \mathbb{C}^{m,1}$ such that 
$Ax$ is “as close as possible” to $b$. Using the Moore--Penrose inverse we 
obtain the best possible approximation with respect to the Euclidean norm.

\vspace{1cm}

\subsubsection*{The Moore-Penrose inverse}

\begin{enumerate}
    \item \textbf{Existence}
    \begin{itemize}
        \item \emph{Usual inverse $A^{-1}$:} Exists only if $A$ is square ($n \times n$) and nonsingular (i.e.\ $\det(A) \neq 0$).
        \item \emph{Moore--Penrose inverse $A^{\dagger}$:} Always exists for any matrix $A$, whether square or rectangular, singular or nonsingular.
    \end{itemize}

    \item \textbf{Defining properties}
    \begin{itemize}
        \item \emph{Usual inverse:} Defined by
        \[
        A A^{-1} = A^{-1} A = I.
        \]
        \item \emph{Moore--Penrose inverse:} Defined by the \textbf{four Penrose conditions}:
        \[
        A A^{\dagger} A = A, \quad
        A^{\dagger} A A^{\dagger} = A^{\dagger}, \quad
        (A A^{\dagger})^H = A A^{\dagger}, \quad
        (A^{\dagger} A)^H = A^{\dagger} A.
        \]
        These conditions ensure a well-defined ``best possible'' generalized inverse.
    \end{itemize}

    \item \textbf{Relation to solutions of linear systems}
    \begin{itemize}
        \item \emph{Usual inverse:}  
        If $A$ is invertible, the system $Ax = b$ has the unique solution
        \[
        x = A^{-1} b.
        \]
        \item \emph{Moore--Penrose inverse:}  
        \begin{itemize}
            \item If $Ax=b$ is consistent, then
            \[
            x = A^{\dagger} b
            \]
            gives the solution of minimum norm.
            \item If $Ax=b$ is inconsistent, then
            \[
            x = A^{\dagger} b
            \]
            gives the least-squares solution, i.e.
            \[
            x = \arg\min_{z \in \mathbb{C}^{m,1}} \|Az - b\|_2.
            \]
        \end{itemize}
    \end{itemize}
\end{enumerate}


    
\end{definition}

\vspace{1cm}

\begin{theorem}[Theorem 19.7]

    Let $A \in \mathbb{C}^{n,m}$ with $n \geq m$ and $b \in \mathbb{C}^{n,1}$ be given.  
If $A = V \Sigma W^H$ is an SVD, and $A^{\dagger}$ is as in (19.4), then 
$\hat{x} = A^{\dagger} b$ satisfies
\[
\| b - A\hat{x} \|_2 \;\leq\; \| b - Ay \|_2 
\quad \text{for all } y \in \mathbb{C}^{m,1},
\]
and
\[
\| \hat{x} \|_2
= \left( \sum_{j=1}^r 
\left| \frac{v_j^H b}{\sigma_j} \right|^2
\right)^{\tfrac{1}{2}}
\;\leq\; \| y \|_2
\quad \text{for all } y \in \mathbb{C}^{m,1} 
\text{ with } \| b - A\hat{x} \|_2 = \| b - Ay \|_2 .
\]

    
\end{theorem}

\vspace{3cm}

The minimization problem for the vector $\hat{x}$ can be written as
\[
\| b - A\hat{x} \|_2^2 
= \min_{y \in \mathbb{C}^{m,1}} \| b - Ay \|_2^2.
\]

If
\[
A =
\begin{bmatrix}
\tau_1 & 1 \\
\vdots & \vdots \\
\tau_m & 1
\end{bmatrix}
\in \mathbb{R}^{m,2},
\]
for pairwise distinct $\tau_1, \ldots, \tau_m \in \mathbb{R}$, then this minimization 
problem corresponds to the problem of linear regression and the least-squares 
approximation in Example 12.16, that we have solved with the QR-decomposition 
of $A$.

If $A = QR$ is this decomposition, then
\[
A^{\dagger} = (A^H A)^{-1} A^H 
\quad \text{(cp.\ Exercise 19.5),}
\]
and we have
\[
A^{\dagger} = (R^H Q^H Q R)^{-1} R^H Q^H 
= R^{-1} (R^H)^{-1} R^H Q^H 
= R^{-1} Q^H.
\]

Thus, the solution of the least-squares approximation in Example 12.16 is identical 
to the solution of the above minimization problem using the SVD of $A$.


\newpage

\subsection{Dynamic Mode Decomposition (DMD)}

\vspace{1cm}

The method relies simply on collecting snapshots of data $x_k$ from a dynamical system 
at a number of times $t_k$, where $k = 1,2,3,\dots,m$. As we will show, DMD is 
algorithmically a regression of data onto locally linear dynamics
\[
x_{k+1} = A x_k,
\]
where $A$ is chosen to minimize
\[
\| x_{k+1} - A x_k \|_2
\]
over the $k = 1,2,3,\dots,m-1$ snapshots.  

The cost of the algorithm is a singular 
value decomposition (SVD) of the snapshot matrix constructed from the data $x_k$.

The DMD procedure constructs the proxy, approximate locally linear dynamical system
\[
\frac{d x}{dt} = \mathcal{A}x  ,
\]
with initial condition $x(0)$ and well-known solution 
\[
x(t) = \sum_{k=1}^n \phi_k \, e^{\omega_k t} \, b_k 
= \Phi \, e^{\Omega t} \, b,
\]
where $\phi_k$ and $\omega_k$ are the eigenvectors and eigenvalues of the matrix $\mathcal{A}$, 
and the coefficients $b_k$ are the coordinates of $x(0)$ in the eigenvector basis.  

Given continuous dynamics, it is always possible to describe an analogous 
discrete-time system sampled every $\Delta t$ in time:
\[
x_{k+1} = A x_k,
\qquad \text{where } A = exp{(\mathcal{A} \Delta t)}.
\]

To minimize the approximation error across all snapshots from 
$k = 1,2,\dots,m$, it is possible to arrange the $m$ snapshots into two large data matrices:
\[
X =
\begin{bmatrix}
x_1 & x_2 & \cdots & x_{m-1}
\end{bmatrix},
\]
\[
X' =
\begin{bmatrix}
x_2 & x_3 & \cdots & x_m
\end{bmatrix}.
\]

The locally linear approximation may be written in terms of these data matrices as
\[
X' \approx A X.
\]

The best-fit $A$ matrix is given by
\[
A = X' X^{\dagger},
\]
where ${}^\dagger$ denotes the Moore--Penrose pseudoinverse. This solution minimizes the error
\[
\| X' - A X \|_F,
\]
where $\|\cdot\|_F$ denotes the Frobenius norm.

\vspace{1cm}

\subsubsection*{The DMD Algorithm}

\vspace{1cm}

1. First, take the singular value decomposition (SVD) of $X$ :
\[
X \approx U \Sigma V^*,
\]
where ${}^*$ denotes the conjugate transpose, 
$U \in \mathbb{C}^{n \times r}$, 
$\Sigma \in \mathbb{C}^{r \times r}$, 
and $V \in \mathbb{C}^{m \times r}$.  
Here $r$ is the rank of the reduced SVD approximation to $X$.  
The left singular vectors $U$ are the POD modes.  
The columns of $U$ are orthonormal, so $U^* U = I$; similarly, $V^* V = I$.

---

2. The matrix $A$ from may be obtained using the pseudoinverse of $X$ from the SVD:
\[
A = X' V \Sigma^{-1} U^*.
\]
In practice, it is computationally more efficient to compute $\tilde{A}$, the $r \times r$ 
projection of the full matrix $A$ onto POD modes:
\[
\tilde{A} = U^* A U = U^* X' V \Sigma^{-1}.
\]
The matrix $\tilde{A}$ defines a low-dimensional linear model of the dynamical system 
on POD coordinates:
\[
\tilde{x}_{k+1} = \tilde{A} \tilde{x}_k.
\]
It is possible to reconstruct the high-dimensional state via $x_k = U \tilde{x}_k$.

---

3. Compute the eigendecomposition of $\tilde{A}$:
\[
\tilde{A} W = W \Lambda,
\]
where the columns of $W$ are eigenvectors and $\Lambda$ is a diagonal matrix 
containing the corresponding eigenvalues $\lambda_k$.

---

4. Finally, we may reconstruct the eigendecomposition of $A$ from $W$ and $\Lambda$.  
In particular, the eigenvalues of $A$ are given by $\Lambda$, and the eigenvectors of $A$ 
(the **DMD modes**) are given by the columns of $\Phi$:
\[
\Phi = X' V \Sigma^{-1} W.
\]

\vspace{1cm}

\subsubsection*{The DMD Function in Matlab}



\begin{lstlisting}[language=Matlab, caption={}]
function [Phi, omega, lambda, b, Xdmd] = DMD(X1, X2, r, dt)
% function [Phi ,omega ,lambda ,b,Xdmd ] = DMD (X1 ,X2 ,r,dt)
% Computes the Dynamic Mode Decomposition of X1 , X2
%
% INPUTS :
%   X1 = X, data matrix
%   X2 = X’, shifted data matrix
%   Columns of X1 and X2 are state snapshots
%   r = target rank of SVD
%   dt = time step advancing X1 to X2 (X to X’)
%
% OUTPUTS :
%   Phi   : the DMD modes
%   omega : the continuous-time DMD eigenvalues
%   lambda: the discrete-time DMD eigenvalues
%   b     : a vector of magnitudes of modes Phi
%   Xdmd  : the data matrix reconstructed by Phi, omega, b
%
% DMD algorithm
[U, S, V] = svd(X1, 'econ');
r = min(r, size(U,2));
U_r = U(:, 1:r);        % truncate to rank-r
S_r = S(1:r, 1:r);
V_r = V(:, 1:r);

Atilde = U_r' * X2 * V_r / S_r;     % low-rank dynamics
[W_r, D] = eig(Atilde);
Phi = X2 * V_r / S_r * W_r;         % DMD modes

lambda = diag(D);                   % discrete-time eigenvalues
omega = log(lambda) / dt;           % continuous-time eigenvalues

% Compute DMD mode amplitudes
x1 = X1(:, 1);
b = Phi \ x1;

% DMD reconstruction
mm1 = size(X1, 2);                  % mm1 = m - 1
time_dynamics = zeros(r, mm1);
t = (0:mm1-1) * dt;                 % time vector

for iter = 1:mm1
    time_dynamics(:, iter) = (b .* exp(omega * t(iter)));
end

Xdmd = Phi * time_dynamics;
\end{lstlisting}

\vspace{1cm}

I started with a basic MATLAB implementation of the Dynamic Mode Decomposition (DMD) algorithm, with the function 

\[
\texttt{[Phi, omega, lambda, b, Xdmd] = DMD(X1, X2, r, dt)}
\]  
  

The algorithm computes an SVD of \(X_1\), truncates it to rank \(r\), and builds the reduced operator  

\[
\tilde{A} = U_r^\top X_2 V_r S_r^{-1}.
\]  

From this reduced operator, it computes the eigenvalues and eigenvectors, constructs the DMD modes, and finally reconstructs the data matrix \(X_{\text{dmd}}\).  

\bigskip
To test this implementation, I created random data using a random stable system. 
I generated a system matrix \(A \in \mathbb{R}^{n \times n}\) with small random entries, shifted by a negative diagonal to ensure stability. 
Starting from a random initial condition \(x_0\), I simulated the system dynamics  

\[
x_{k} = A x_{k-1}, \qquad k=2,\dots,m,
\]  

to build a snapshot matrix \(X = [x_1, x_2, \dots, x_m]\). 
I then split this into \(X_1\) and \(X_2\).  

\bigskip

I introduced three error metrics:  

\[
\text{err}_{\tilde{A}} = \frac{\|U_r^\top A U_r - \tilde{A}\|_F}{\|U_r^\top A U_r\|_F}, \qquad
\text{err}_{X} = \frac{\|X_2 - A_{\text{approx}} X_1\|_F}{\|X_2\|_F}, \qquad
\text{err}_{\text{recon}} = \frac{\|X_1 - X_{\text{dmd}}\|_F}{\|X_1\|_F}.
\]  

Here, 
\(\text{err}_{\tilde{A}}\) measures how well the reduced operator \(\tilde{A}\) approximates the projection of \(A\), 
\(\text{err}_{X}\) evaluates how well the approximate dynamics reproduce the one-step evolution of the snapshots, 
and \(\text{err}_{\text{recon}}\) quantifies the accuracy of the full DMD reconstruction.  


\vspace{1cm}

\begin{lstlisting}[language=Matlab, caption={Main code}]
    % Dimensions
n = 1000;      % state dimension (rows)
m = 50;       % number of snapshots (columns)
dt = 0.1;     % timestep

% Create a random stable system matrix A
A = randn(n,n) * 0.05;        % small random entries
A = A - 0.1*eye(n);           % shift eigenvalues left -> stability

% Initial condition
x0 = randn(n,1);

% Generate snapshot matrix X
X = zeros(n,m);
X(:,1) = x0;
for k = 2:m
    X(:,k) = A * X(:,k-1);   % linear evolution
end

% Build X1 and X2 (shifted matrices)
X1 = X(:,1:end-1);   % size n x (m-1)
X2 = X(:,2:end);     % size n x (m-1)

% Example: run your DMD
r = 5;   % target rank (choose smaller than m-1)
[Phi, omega, lambda, b, Xdmd, Atilde, U_r] = DMD(X1, X2, r, dt);
disp(A);
A_approx = U_r * Atilde * U_r';

% I tried to capture the difference between 
% the actual and the created matrice. But it does 
% not make sense because it captures the
% dominant dynamics along the trajectories I sampled.

% diff_norm = norm(A - A_approx, 'fro');
% rel_err = diff_norm / norm(A, 'fro');
% disp(rel_err);


Ar_sub = U_r' * A * U_r;
err_Atilde = norm(Ar_sub - Atilde, 'fro') / max( eps, norm(Ar_sub,'fro') );
err_X = norm(X2 - A_approx * X1, 'fro') / max( eps, norm(X2,'fro') );
err_recon = norm(X1 - Xdmd, 'fro') / max( eps, norm(X1,'fro') );

fprintf('err_Atilde (reduced-op): %.4e\n', err_Atilde);
fprintf('err_X (data action):      %.4e\n', err_X);
fprintf('err_recon (reconstruction):%.4e\n', err_recon);

\end{lstlisting}


\vspace{1cm}

When computing the Singular Value Decomposition (SVD) in MATLAB, the command
\[
[U, S, V] = \texttt{svd}(X1,'econ')
\]
returns the so-called economy-size decomposition. This means that if $X_1 \in \mathbb{R}^{n \times m}$, then the matrices have dimensions
\[
U \in \mathbb{R}^{n \times p}, \quad S \in \mathbb{R}^{p \times p}, \quad V \in \mathbb{R}^{m \times p}, \quad p = \min(n,m).
\]
By contrast, the full SVD without the option \texttt{'econ'} would return
\[
U \in \mathbb{R}^{n \times n}, \quad S \in \mathbb{R}^{n \times m}, \quad V \in \mathbb{R}^{m \times m}.
\]

In the context of Dynamic Mode Decomposition, only the first $p$ singular vectors are ever used, and subsequently only the leading $r \leq p$ components are retained. Therefore, computing the full SVD produces a large number of unnecessary vectors and zero singular values. This increases both memory usage and computational cost without providing any benefit.

As an illustration, if $X_1$ is a $1000 \times 50$ matrix (that is, a state of dimension $1000$ with $50$ snapshots), then the full SVD produces $U$ of size $1000 \times 1000$, $S$ of size $1000 \times 50$, and $V$ of size $50 \times 50$. On the other hand, the economy SVD produces $U$ of size $1000 \times 50$, $S$ of size $50 \times 50$, and $V$ of size $50 \times 50$. The difference is very significant: instead of storing one million entries in $U$, the economy decomposition requires only fifty thousand.

For this reason, the DMD algorithm almost always uses the economy SVD. It provides exactly the same information that is needed for the reduced-order model, while being substantially more efficient in terms of memory and computational cost.


\vspace{1cm}

The DMD algorithm presented here takes advantage of low dimensionality in the data to make a low-rank approximation of the linear mapping that best approximates the nonlinear dynamics of the data collected for the system. Once this is done,a prediction of the future state of the system is achieved for all time.

\vspace{1cm}

\subsubsection*{The Limitations of the DMD method}

\vspace{1cm}

\begin{itemize}
    \item Translational and rotational invariances

\vspace{1cm}

Example : In this case, one of the signals is translating at a constant velocity across 
the spatial domain. The two signals of interest are
\[
f(x,t) = f_1(x,t) + f_2(x,t)
= \operatorname{sech}(x+6-t)\, e^{i 2.3 t}
+ 2\,\operatorname{sech}(x)\,\tanh(x)\, e^{i 2.8 t}.
\]

The rank-2 reconstruction is no longer able to characterize the dynamics due to the translation. Indeed, it now takes nearly 10 modes to produce an accurate reconstruction. This artificial inflation of the dimension is a result of the inability of SVD to capture translational invariances and correlate across snapshots of time.

\vspace{1cm}

    \item Transient time behavior

\vspace{1cm}

To demonstrate the impact of the transient time dynamics, we once again consider 
the simple example of two mixed spatiotemporal signals, with one turning on and off 
during the time domain. In this case, the signals are
\[
f(x,t) = f_1(x,t) + f_2(x,t)
= 0.5\,\operatorname{sech}(x+5)\, e^{i 2.3 t}\,
\bigl[\tanh(t-\pi) - \tanh(t-3\pi)\bigr]
+ 2\,\operatorname{sech}(x)\,\tanh(x)\, e^{i 2.8 t}.
\]


Interestingly, regardless of the rank (r = 10 in the figure), DMD is incapable of correctly getting the right turn-on and turn-off behavior of the first mode. The result shown is the same for any rank approximation above two. Unlike the translational or rotational invariance, which simply created an artificially high dimension in DMD, in this case DMD completely fails to characterize the correct dynamics

\vspace{1cm}

The discussion of multiresolution DMD (mrDMD) presents a strategy to handle those issues.

\end{itemize}

\vspace{1cm}

\subsubsection*{Fluid Dynamics}

\vspace{1cm}

Since its introduction, DMD has become a widely used technique in fluid dynamics,
with strong connections to Koopman theory and nonlinear dynamical systems.

\begin{itemize}
    \item DMD to extract structure from fluids data
    
DMD has been applied to diverse flow geometries (jets, cavity flows, wakes,
boundary layers, etc.) to analyze mixing, acoustics, and combustion. 
It has revealed coherent modes, oscillations, and instabilities in many flow settings.

    \item Model reduction and system identification
    
DMD provides reduced-order models (ROMs) from simulations and experiments.
It is related to balanced POD and the Eigensystem Realization Algorithm (ERA),
and has been implemented in software such as MODRED.

    \item Control-oriented methods
    
DMD-based approaches support control design. They connect with balanced
truncation, BPOD, and ERA, enabling data-driven input-output models without
adjoint simulations.

    \item Innovations of DMD in fluid dynamics
    
Variants of DMD have been developed, including compressed sensing DMD,
optimal mode decomposition, mode selection, and the use of images or spatial
variables in place of time.

    \item Relationship to the Perron--Frobenius operator
    
DMD is linked to the Koopman and Perron--Frobenius operators, enabling the
identification of almost-invariant sets. These are useful for analyzing coherence,
sensitivity, and uncertainty quantification in fluid systems.
\end{itemize}

\vspace{1cm}

\subsubsection*{Multiresolution Dynamic Mode Decomposition}

\vspace{1cm}

\vspace{1cm}

\subsubsection*{DMD with Control}

\vspace{1cm}

The goal of DMDc is to characterize the relationship between three measurements:
the current measurement $x_k$, the future state $x_{k+1}$, and the current control
$u_k$. The relationship can be approximated by the canonical discrete linear
dynamical system
\[
    x_{k+1} \approx A x_k + B u_k.
\]

The goal is to find approximations of $A$ and $B$ from measurements. 
The state snapshot measurement matrices $X$ and $X'$ are constructed in the same
manner as for DMD. A new measurement snapshot matrix is constructed for the inputs
\[
    \Upsilon =
    \begin{bmatrix}
        \,| & | &  & |\, \\
        u_1 & u_2 & \cdots & u_{m-1} \\
        \,| & | &  & |\, 
    \end{bmatrix}.
\]
The first equation can be rewritten in matrix form to include the new data matrices:
\[
    X' \approx A X + B \Upsilon.
\]

DMDc utilizes the three data matrices X, X' and $ \Upsilon $ to discover a best-fit approximation of operators A and B. The algorithm is different whether the matrix B is known (or well-estimated) or unknown.

\vspace{1cm}

\subsubsection*{The DMDc Algorithm}

\vspace{1cm}

\begin{enumerate}
    \item \textbf{Collect and construct the snapshot matrices.}  
    
    Collect the system measurement and control snapshots and form the matrices 
    $X$, $X'$, and $\Upsilon$. Stack the data matrices $X$ and $\Upsilon$ to 
    construct the matrix $\Omega$.

    \item \textbf{Compute the SVD of the input space $\Omega$.}  
    
    Compute the SVD of $\Omega$ to obtain the decomposition
    \[
        \Omega \approx \tilde{U} \tilde{\Sigma} \tilde{V}^*
    \]
    with truncation value $\tilde{r}$.

    \item \textbf{Compute the SVD of the output space $X'$.}  
    
    Compute the SVD of $X'$ to obtain the decomposition
    \[
        X' \approx \hat{U} \hat{\Sigma} \hat{V}^*
    \]
    with truncation value $r$.

    \item \textbf{Compute the approximation of the operators $G = [A \; B]$.}  
    \begin{align}
        \tilde{A} &= \hat{U}^* X' \tilde{V} \tilde{\Sigma}^{-1} \tilde{U}_1^* \hat{U}, \\
        \tilde{B} &= \hat{U}^* X' \tilde{V} \tilde{\Sigma}^{-1} \tilde{U}_2^*.
    \end{align}

    \item \textbf{Perform the eigenvalue decomposition of $\tilde{A}$.}  
    \[
        \tilde{A} W = W \Lambda.
    \]

    \item \textbf{Compute the dynamic modes of the operator $A$.}  
    \[
        \Phi = X' \tilde{V} \tilde{\Sigma}^{-1} \tilde{U}_1^* \hat{U} W.
    \]
\end{enumerate}

\vspace{1cm}

\subsubsection*{The DMDc Matlab Algorithm}

\vspace{1cm}

\begin{lstlisting}[language=Matlab, caption={}]
X   = StateData(:, 1:end-1);
Xp  = StateData(:, 2:end);
Ups = InputData(:, 1:end-1);
Omega = [X; Ups];

[U, Sig, V] = svd(Omega, 'econ');
thresh = 1e-10;
rtil = length(find(diag(Sig) > thresh));
Util = U(:, 1:rtil);
Sigtil = Sig(1:rtil, 1:rtil);
Vtil = V(:, 1:rtil);

[U, Sig, V] = svd(Xp, 'econ');
thresh = 1e-10;
r = length(find(diag(Sig) > thresh));
Uhat = U(:, 1:r);
Sighat = Sig(1:r, 1:r);
Vbar = V(:, 1:r);

n = size(X, 1);
q = size(Ups, 1);
U_1 = Util(1:n, :);
U_2 = Util(n+q:n+q, :);
approxA = Uhat' * Xp * Vtil * inv(Sigtil) * U_1' * Uhat;
approxB = Uhat' * Xp * Vtil * inv(Sigtil) * U_2';

[W, D] = eig(approxA);

Phi = Xp * Vtil * inv(Sigtil) * U_1' * Uhat * W;
\end{lstlisting}

\vspace{1cm}

\subsubsection*{Example}

\vspace{1cm}

DMDc can be utilized to discover the underlying dynamics of an unstable system that
has a stabilizing controller. Without the inclusion of measurements of the inputs,
DMD would produce a set of stable eigenvalues and dynamic modes. DMDc, though,
can recover the unstable dynamics. Consider the unstable linear dynamical system
\[
    \begin{bmatrix} x_1 \\ x_2 \end{bmatrix}_{k+1}
    =
    \begin{bmatrix} 1.5 & 0 \\ 0 & 0.1 \end{bmatrix}
    \begin{bmatrix} x_1 \\ x_2 \end{bmatrix}_k
    +
    \begin{bmatrix} 1 \\ 0 \end{bmatrix} u_k, 
\]
where $u_k = K[x_1]_k$ and $K=-1$.  
The proportional controller shifts the unstable eigenvalue, $\lambda = 1.5$, 
to a stable value $\lambda = 0.5$ within the unit circle. 
Recall that eigenvalues outside the unit circle in the complex plane are unstable 
for discrete-time systems.  

We can produce artificial data from this model using an initial condition
$x_0 = [4 \; 7]^T$. The first five temporal snapshots are
\[
X =
\begin{bmatrix}
4 & 2 & 1 & 0.5 \\
7 & 0.7 & 0.07 & 0.007
\end{bmatrix},
\]
\[
X' =
\begin{bmatrix}
2 & 1 & 0.5 & 0.25 \\
0.7 & 0.07 & 0.007 & 0.0007
\end{bmatrix},
\]
\[
\Upsilon =
\begin{bmatrix}
-4 & -2 & -1 & -0.5
\end{bmatrix}.
\]

Utilizing these data matrices and a known input matrix $B$, the description of DMDc 
in §6.1.2 can be applied.  
The SVD of $X$ gives the factorization
\[
U =
\begin{bmatrix}
-0.5239 & -0.8462 \\
-0.8462 & 0.5329
\end{bmatrix},
\quad
\Sigma =
\begin{bmatrix}
8.2495 & 0 \\
0 & 1.6402
\end{bmatrix},
\]
\[
V =
\begin{bmatrix}
-0.9764 & 0.2105 \\
-0.2010 & -0.8044 \\
-0.0718 & -0.4932 \\
-0.0330 & -0.2557
\end{bmatrix}. 
\]

From these, and the matrix $B$, we can compute an approximation to $A$:
\[
    \bar{A} =
    \begin{bmatrix}
    1.5 & 0 \\
    0 & 0.1
    \end{bmatrix},
\]
thus recovering the unstable dynamics from data.  
DMD alone would incorrectly yield a stable operator, whereas DMDc correctly
disambiguates the dynamics.

\bigskip
\noindent\textbf{ALGORITHM 6.1. DMDc for unstable systems with known B matrix}
\begin{lstlisting}[language=Matlab, caption={}]
% Data collection
A = [1.5 0; 0 0.1];
x0 = [4; 7];
K  = [-1];
m  = 20;
DataX = x0;
DataU = [0];
B = [1; 0];

for j = 1:m
    DataX(:, j+1) = A * DataX(:, j) + B .* (K * DataX(:, j));
    DataU(:, j)   = K .* DataX(1, j);
end

% Data matrices
X   = DataX(:, 1:end-1);
Xp  = DataX(:, 2:end);
Ups = DataU;

% SVD
[U, Sig, V] = svd(X, 'econ');
thresh = 1e-10;
r = length(find(diag(Sig) > thresh));
U   = U(:, 1:r);
Sig = Sig(1:r, 1:r);
V   = V(:, 1:r);

% DMD
A_DMD = Xp * V * inv(Sig) * U';

% DMDc - B is known
A_DMDc = (Xp - B * Ups) * V * inv(Sig) * U';
\end{lstlisting}


\newpage

\subsection{Operator Inference}

\vspace{1cm}

We consider here projection-based model reduction for full models that are based on parametrized time-dependent partial differential equations (PDEs). Projection-based model reduction first constructs a basis of a low-dimensional reduced space and then projects the equations of the full model onto the reduced space to obtain the operators of the reduced model.

We present nonintrusive
operator inference that replaces the classical intrusive reduced operator construction by deriving approximations of the reduced operators directly from data of the full model, without requiring the full-model operators. The data include initial conditions, inputs, trajectories of the full-model states, and outputs. The reduced operators are low-dimensional quantities and therefore the inference underlying our approach is feasible with respect to computational costs and with respect to required amount of data. Our operator inference is applicable to nonlinear PDEs with polynomial nonlinear terms of low order. Our operator inference provides a nonintrusive way to construct a reduced model.

Our operator inference follows a similar approach to DMD. We also optimize for operators that best fit the trajectories in the L2 norm. In the case of a linear full model and inputs that are constant over time, our inferred operator is the same operator as the one obtained via DMD; however, DMD derives a linear approximation of the full model, whereas our approach can handle full models that have polynomial nonlinear terms. Note that a first approach exists to extend DMD to nonlinear full models via the Koopman operator that transforms a finite-dimensional nonlinear system into an infinite-dimensional linear system.

\vspace{1cm}

\subsubsection*{Algorithm  for models with second-order nonlinear terms}

\vspace{1cm}

\begin{algorithm}
\begin{algorithmic}[1]
\Procedure{INFEROP}{$x_0, X, U, Y, V_n$}
    \State Project $\hat{x}_0 = V_n^T x_0$ and $\hat{X} = X V_n$
    \State Assemble matrices $\hat{X}^{(1)}, \dots, \hat{X}^{(n)}$ following (11)
    \State Assemble data matrix $D$ following (14): 
          \[
             D = [\hat{X}, U, \hat{X}^{(1)}, \dots, \hat{X}^{(n)}] \in \mathbb{R}^{K \times (n+p+s)}
          \]
    \State Assemble right-hand side matrix $R$ using (24): 
          \[
             R = 
             \begin{bmatrix}
                \frac{\hat{x}_1 - \hat{x}_0}{\delta t}, \dots, 
                \frac{\hat{x}_K - \hat{x}_{K-1}}{\delta t}
             \end{bmatrix}^T 
             \in \mathbb{R}^{K \times n}
          \]
    \State Extract columns $R = [r_1, \dots, r_n]$ and solve for $O^T = [o_1, \dots, o_n]$ in
          \[
             \min \|D o_i - r_i\|_2^2, \quad i = 1, \dots, n
          \]
    \State Extract operators $\hat{A}, \hat{B}, \hat{F}^{(1)}, \dots, \hat{F}^{(n)}$ from $O$ as in (19)
    \State Assemble $\hat{F} = [\hat{F}^{(1)}, \dots, \hat{F}^{(n)}]$
    \State Solve least-squares problem (23) to construct matrix $\hat{C}$
    \State \Return $\hat{A}, \hat{B}, \hat{C}, \hat{F}$
\EndProcedure
\end{algorithmic}
\end{algorithm}

\vspace{1cm}


\subsubsection*{Problem setup}

We consider systems of parametrized nonlinear ordinary differential equations (ODEs) of the form
\[
\frac{d}{dt} x(t;\mu) = f(x(t;\mu), u(t;\mu); \mu),
\qquad x(0;\mu) = x_0(\mu),
\]
with state vector $x(t;\mu) \in \mathbb{R}^N$, input $u(t;\mu) \in \mathbb{R}^p$, 
parameter $\mu \in \mathcal{D} \subset \mathbb{R}^d$, and output
\[
y(t;\mu) = C(\mu)x(t;\mu) \in \mathbb{R}^q.
\]

A common case of interest is when the dynamics include quadratic nonlinearities:
\[
\frac{d}{dt} x(t;\mu) = A(\mu)x(t;\mu) + F(\mu)x(t;\mu)^2 + B(\mu)u(t;\mu),
\qquad y(t;\mu) = C(\mu)x(t;\mu),
\]
where $A(\mu) \in \mathbb{R}^{N \times N}$ corresponds to the linear term, 
$F(\mu) \in \mathbb{R}^{N \times S}$ to quadratic interactions, and 
$B(\mu), C(\mu)$ to the input and output operators.  
Here $x(t;\mu)^2 \in \mathbb{R}^S$ collects all quadratic monomials of the components of $x(t;\mu)$, 
with $S = N(N+1)/2$.  

\vspace{1cm}

\subsubsection*{Time discretization}

Let $0 = t_0 < t_1 < \cdots < t_K = T$ be the discretization of the time domain $[0,T]$ into 
$K \in \mathbb{N}$ equidistant time steps with step size $0 < \delta t \in \mathbb{R}$.  
Let $x_1(\mu), \dots, x_K(\mu) \in \mathbb{R}^N$ be the discrete states of the ODE system at 
times $t_1,\dots,t_K$, computed with a time-stepping scheme, and let 
$y_1(\mu), \dots, y_K(\mu) \in \mathbb{R}^q$ be the corresponding outputs.  

We define the trajectory
\[
X(\mu) = 
\begin{bmatrix}
x_1(\mu) \\
\vdots \\
x_K(\mu)
\end{bmatrix}
\in \mathbb{R}^{K \times N},
\qquad
Y(\mu) = 
\begin{bmatrix}
y_1(\mu) \\
\vdots \\
y_K(\mu)
\end{bmatrix}
\in \mathbb{R}^{K \times q},
\]
and the input matrix
\[
U(\mu) = 
\begin{bmatrix}
u(t_1;\mu) \\
\vdots \\
u(t_K;\mu)
\end{bmatrix}
\in \mathbb{R}^{K \times p}.
\]

The trajectory $X(\mu)$ and the output matrix $Y(\mu)$ contain the discrete states and outputs 
derived with a time-stepping scheme, while $U(\mu)$ contains the inputs of the continuous model 
sampled at the same discrete time steps.  
We assume that the scheme is convergent, i.e. as $\delta t \to 0$, the discrete states $x_j(\mu)$ 
converge in the $L^2$ sense to the continuous solution $x(t_j;\mu)$.  

\vspace{1cm}


\subsubsection*{Data projection}

Let $V_n \in \mathbb{R}^{N \times n}$ be the POD basis matrix containing as columns the first $n$ POD basis vectors.  
The projected state $\hat{x}(t) \in \mathbb{R}^n$ of the full-model state $x(t) \in \mathbb{R}^N$ is
\[
\hat{x}(t) = V_n^T x(t).
\]

For the discrete states $x_1, \dots, x_K$, this gives
\[
\hat{x}_j = V_n^T x_j, \qquad j = 1,\dots,K,
\]
which we assemble into the projected trajectory
\[
\hat{X} = 
\begin{bmatrix}
\hat{x}_1 \\
\vdots \\
\hat{x}_K
\end{bmatrix}
\in \mathbb{R}^{K \times n}.
\]

We also define quadratic lifted data from the projected states:
\[
\hat{X}^{(i)} =
\begin{bmatrix}
\hat{x}_1^{(i)} \\
\vdots \\
\hat{x}_K^{(i)}
\end{bmatrix}
\in \mathbb{R}^{K \times i},
\qquad i=1,\dots,n,
\]
where each $\hat{x}_j^{(i)}$ contains the $i$-th quadratic combinations of $\hat{x}_j$



\vspace{1cm}


\subsubsection*{Inference of reduced operators}

\vspace{1cm}

We now infer the reduced operators directly from the projected data. 
The goal is to approximate the dynamics of the reduced states without 
requiring access to the full-model operators.

Considering the previous discrete time steps, we seek reduced operators $\hat{A} \in \mathbb{R}^{n \times n}$, 
$\hat{B} \in \mathbb{R}^{n \times p}$, and 
$\hat{F} \in \mathbb{R}^{n \times s}$ 
that minimize the difference between the time derivatives 
and the reduced-order dynamics. This leads to the optimization problem
\begin{equation}
    \min_{\hat{A}, \hat{B}, \hat{F}} 
    \sum_{j=1}^K \big\| \dot{\hat{x}}_j - \hat{A}\hat{x}_j - \hat{F}\hat{x}_j^2 - \hat{B}u_j \big\|_2^2.
\end{equation}


To express the problem compactly, we define the data matrix
\begin{equation}
    D = [\hat{X}, U, \hat{X}^{(1)}, \dots, \hat{X}^{(n)}] \in \mathbb{R}^{K \times (n+p+s)},
\end{equation}
where $\hat{X}$ is the projected trajectory, $U$ is the input matrix, 
and $\hat{X}^{(i)}$ are the matrices derived from the projected states 
as in Eq.~(11). The right-hand side matrix is
\begin{equation}
    R = [\dot{\hat{x}}_1, \dots, \dot{\hat{x}}_K]^T \in \mathbb{R}^{K \times n}.
\end{equation}

The optimization problem can now be written as a linear least-squares problem:
\begin{equation}
    \min_{O \in \mathbb{R}^{n \times (n+p+s)}} \| D O^T - R \|_F^2,
\end{equation}
where
\[
    O = [\hat{A}, \hat{B}, \hat{F}^{(1)}, \dots, \hat{F}^{(n)}].
\]
This problem decouples into $n$ independent least-squares problems, 
one for each column of $R$, which can be solved efficiently.

\vspace{1cm}


\vspace{1cm}


\vspace{1cm}

\vspace{1cm}

\vspace{1cm}

\vspace{1cm}

\vspace{1cm}

\vspace{1cm}

\end{document}
